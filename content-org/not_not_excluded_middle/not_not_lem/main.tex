\documentclass{article}

\usepackage{symbols}

\input{kldb}
\makeindex
\makeglossaries

\begin{document}

\title{Not not Law of Excluded Middle and Glivenko's translation}
\author{Jose Abel Castellanos Joo\\Department of Computer Science\\University
of New Mexico\\}

\date{\today}
\maketitle
%\tableofcontents

In intuitionistic logic, the intuitionistic negation $\neg A$ is encoded as $A
\rightarrow \bot$. 

\section{Theorems}

\begin{theorem}
  $\vdash_I \neg\neg(A \lor \neg A)$
\end{theorem}

\begin{proof}

 \begin{prooftree}
  \infer0[1]{A}
  \infer1[Left intro $\lor$]{A \lor (A \rightarrow \bot)}
  \infer0[2]{(A \lor (A \rightarrow \bot)) \rightarrow \bot}
  \infer2[MP]{\bot}
  \infer1[Intro-1]{A \rightarrow \bot}
  \infer1[Right intro $\lor$]{A \lor (A \rightarrow \bot)}
  \infer0[2]{(A \lor (A \rightarrow \bot)) \rightarrow \bot}
  \infer2[MP]{\bot}
  \infer1[Intro-2]{((A \lor (A \rightarrow \bot)) \rightarrow \bot) \rightarrow
    \bot} 
\end{prooftree} 

\end{proof}

\begin{theorem}
  If $\vdash_I \neg \neg A$ and $\vdash_I \neg \neg (A \rightarrow B)$,then
  $\vdash_I \neg \neg B$ 
\end{theorem}

\begin{proof}

  \begin{prooftree}
  \infer0[1]{A}
  \infer0[2]{A \rightarrow B}
  \infer2[MP]{B}
  \infer0[3]{B \rightarrow \bot}
  \infer2[MP]{\bot}
  \infer1[Intro-1]{A \rightarrow \bot}
  \infer0[Hyp]{(A \rightarrow \bot) \rightarrow \bot}
  \infer2[MP]{\bot}
  \infer1[2]{(A \rightarrow B) \rightarrow \bot}
  \infer0[Hyp]{((A \rightarrow B) \rightarrow \bot) \rightarrow \bot}
  \infer2[MP]{\bot}
  \infer3[Intro-3]{(B \rightarrow \bot) \rightarrow \bot}
\end{prooftree}

\end{proof}

%\bibliographystyle{plain}
%\bibliography{./../../../references}
%\printglossaries
%\printindex
\end{document}

